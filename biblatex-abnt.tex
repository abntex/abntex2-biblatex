\documentclass[a4paper]{article}
\usepackage[brazil]{babel}
\usepackage{lmodern}
\usepackage{microtype}
\usepackage{enumitem}
\usepackage{examplep}
\usepackage{hyperref}
\usepackage[T1]{fontenc}
\usepackage[utf8]{inputenc}

\usepackage{parskip}
\setlength{\parskip}{1.2\baselineskip}

\usepackage{titlesec}
\newcommand{\subsectionbreak}{\clearpage}

\usepackage[
    backend=biber,
    style=abnt,
    %hyperref,         % Uncomment to turn citations into links
    %backref,          % Uncomment for back references (E.g. ``Cit. on p. 1")
    %repeatfields,     % Uncomment to repeat fields instead of using an underscore
    %usedashes,        % Uncomment to use biblatex standard dashes instead of underscores
    giveninits,       % Uncomment to use initials for first names
    uniquename=init,   % When using giveninits only initials can be used for disambiguation
    %bftitles,         % Uncomment to print titles in bold
    %indent,           % Uncomment to use hanging indentation in the bibliography
    %scbib,            % Uncomment to use small caps in the bibliography
    %sccite,           % Uncomment to use small caps in the citations
    %noslsn,           % Uncomment to hide [s.l], [s.n] and [s.l.: s.n.]
    %nosl,             % Uncomment to hide just [s.l]
    %nosn,             % Uncomment to hide just [s.n]
]{biblatex}
\usepackage[brazil]{babel}
\usepackage[autostyle]{csquotes}

\addbibresource{biblatex-abnt.bib}

\DeclareBibliographyCategory{singleentries}

\newcommand{\singlecite}[1]{%
  \addtocategory{singleentries}{#1}%
  \defbibcheck{key#1}{
    \iffieldequalstr{entrykey}{#1}
      {}
      {\skipentry}}%
  \printbibliography[heading=none,check=key#1]%
}

\title{biblatex-abnt}
\author{Daniel B. Marques}

\begin{document}

\maketitle

\tableofcontents

\clearpage
\section{Requisitos}

O \texttt{biblatex-abnt} requer \texttt{biblatex v3.3} e \texttt{biber v2.4}. Caso haja algum problema na compilação, cheque se seus pacotes estão atualizados.

\section{Instalação}

O \texttt{biblatex-abnt} está incluso no TeX Live 2016.

Para instalá-lo manualmente, copie os arquivos \texttt{.bbx}, \texttt{.cbx} e \texttt{.lbx} para:
\begin{verbatim}
    <TEXMFLOCAL>/tex/latex/biblatex-contrib/biblatex-abnt/
\end{verbatim}

\section{Uso}

Para usar o {biblatex-abnt}, adicione as seguintes linhas ao preâmbulo do seu arquivo {.tex}:

\begin{verbatim}
    \usepackage[backend=biber, style=abnt]{biblatex}
    \usepackage[brazil]{babel}
    \addbibresource{arquivo.bib}        % Seus arquivos de
    \addbibresource{outroarquivo.bib}   % bibliografia vão aqui
\end{verbatim}

Após as opções \texttt{backend=biber} e \texttt{style=abnt}, podem ser acrescentadas as opções descritas na seção \ref{sec:opções}.

Use os comandos descritos na seção \ref{sec:comandos} para citar obras.

Use o comando \verb"\printbibliography" para imprimir a bibliografia.

\clearpage
\section{Comandos}
\label{sec:comandos}

\begin{description}[style=nextline]
    \item [\PVerb{\cite{bosi08}}] \cite{bosi08}
    \item [\PVerb{\textcite{bosi08}}] \textcite{bosi08}
    \item [\PVerb{\cite*{bosi08}}] \cite*{bosi08}
    \item [\PVerb{\textcite*{bosi08}}] \textcite*{bosi08}
    \item [\PVerb{\cites{mann09}{moretti09:1}{moretti09}}] \cites{mann09}{moretti09:1}{moretti09}
    \item [\PVerb{\cites{mann09}{moretti09:1, moretti09}}] \cites{mann09}{moretti09:1, moretti09}
    \item [\PVerb{\textcites{moretti09}{mann09}{amaral15}}] \textcites{moretti09}{mann09}{amaral15}
    \item [\PVerb{\apud{assis08}{bosi08}}] \apud{assis08}{bosi08}
    \item [\PVerb{\apud[p.~12]{assis08}[p.~200]{bosi08}}] \apud[p.~12]{assis08}[p.~200]{bosi08}
    \item [\PVerb{\textapud[p.~200]{assis08}[p.~12]{bosi08}}] \textapud[p.~200]{assis08}[p.~12]{bosi08}
    \item [\PVerb{\apud[batman][]{bosi08}}] \apud[batman][]{bosi08}
    \item [\PVerb{Assis \cite[apud][p.~200]{bosi08}}] Assis \cite[apud][p.~200]{bosi08}
\end{description}

\clearpage
\section{Opções}
\label{sec:opções}

As opções a seguir podem ser usadas ao chamar o pacote \texttt{biblatex}:

\begin{description}
    \item [hyperref] Transforma as citações em links que levam à bibliografia
    \item [backref] Aponta, na bibliografia, as páginas em que a entrada foi citada
    \item [repeatfields] Imprime os campos repetidos na bibliografia, em vez de substituí-los por traços sublineares
    \item [usedashes] Usa os traços padrão do \texttt{biblatex} em vez de traços sublineares nos campos repetidos
    \item [giveninits] Abrevia os primeiros nomes na bibliografia
    \item [uniquename=init] Essa opção deve ser usada em conjunto com a opção \texttt{giveninits}
    \item [bftitles] Usa negrito para os títulos na bibliografia
    \item [indent] Indenta as entradas da bibliografia
    \item [scbib] Imprime os nomes em versalete na bibliografia
    \item [sccite] Imprime os nomes em versalete nas citações
    \item [noslsn] Oculta as abreviações [s.l], [s.n] e [s.l.: s.n.] na bibliografia
    \item [nosl] Oculta apenas as abreviações [s.l.]
    \item [nosn] Oculta apenas as abreviações [s.n.]
\end{description}

E.g.: \verb"\usepackage[backend=biber, style=abnt, bftitles]{biblatex}"

As opções \texttt{repeatfields}, \texttt{nosl}, \texttt{nosn} e \texttt{noslsn} também podem ser usadas apenas em entradas específicas. E.g.:

\begin{verbatim}
    @mvbook{assis08,
        author  = {Machado de Assis},
        title   = {Obra completa em quatro volumes},
        year    = {2008},
        options = {repeatfields, noslsn=false}
    }
\end{verbatim}



\clearpage
\section{Entradas comuns}

A lista completa de campos e entradas pode ser encontrada no manual do \texttt{biblatex}. Estes são alguns exemplos de situações comuns:

\begingroup
\let\clearpage\relax
\subsection{@mvbook}
\endgroup

Um livro abrangendo múltiplos volumes:

\begin{verbatim}
    @mvbook{assis08,
        author     = {Machado de Assis},
        title      = {Obra completa em quatro volumes},
        year       = {2008},
        editor     = {Aluizio Leite and
                      Ana Lima Cecilio and Heloisa Jahn},
        editortype = {organizer},
        edition    = {2},
        volumes    = {4},
        publisher  = {Nova Fronteira},
        location   = {Rio de Janeiro},
        series     = {Biblioteca luso-brasileira.
                      Série brasileira},
    }
\end{verbatim}

\singlecite{assis08}

\subsection{@book}

Um único livro. Pode ser um dos volumes de um livro que abrange múltiplos volumes:

\begin{verbatim}
    @book{assis08:1,
        volume     = {1},
        title      = {Fortuna crítica/Romance},
        pagetotal  = {1340},
        author     = {Machado de Assis},
        maintitle  = {Obra completa em quatro volumes},
        year       = {2008},
        editor     = {Aluizio Leite and
                      Ana Lima Cecilio and Heloisa Jahn},
        editortype = {organizer},
        edition    = {2},
        publisher  = {Nova Fronteira},
        location   = {Rio de Janeiro},
        series     = {Biblioteca luso-brasileira.
                      Série brasileira},
    }
\end{verbatim}

\noindent
Também é possível usar o campo \texttt{crossref} para herdar as informações de outra entrada:

\begin{verbatim}
    @book{assis08:1,
        crossref  = {assis08},
        volume    = {1},
        title     = {Fortuna crítica/Romance},
        pagetotal = {1340},
    }
\end{verbatim}

\singlecite{assis08:1}

\subsection{@bookinbook}

Uma obra originalmente publicada por si só, mas citada como parte de outro livro:

\begin{verbatim}
    @bookinbook{assis08:1b,
        title      = {Esaú e Jacó},
        pages      = {1073-1226},
        volume     = {1},
        booktitle  = {Fortuna crítica/Romance},
        pagetotal  = {1340},
        author     = {Machado de Assis},
        maintitle  = {Obra completa em quatro volumes},
        year       = {2008},
        editor     = {Aluizio Leite and
                      Ana Lima Cecilio and Heloisa Jahn},
        editortype = {organizer},
        edition    = {2},
        publisher  = {Nova Fronteira},
        location   = {Rio de Janeiro},
        series     = {Biblioteca luso-brasileira.
                      Série brasileira},
    }
\end{verbatim}

\noindent
Ou:

\begin{verbatim}
    @bookinbook{assis08:1b,
        crossref = {assis08:1},
        title    = {Esaú e Jacó},
        pages    = {1073-1226},
    }
\end{verbatim}

\singlecite{assis08:1b}

\subsection{@inbook}

Uma parte de um livro que forma uma unidade independente, com seu próprio título:

\begin{verbatim}
    @inbook{bosi08,
        title      = {Uma figura machadiana},
        author     = {Alfredo Bosi},
        pages      = {179-189},
        volume     = {1},
        booktitle  = {Fortuna crítica/Romance},
        pagetotal  = {1340},
        bookauthor = {Machado de Assis},
        maintitle  = {Obra completa em quatro volumes},
        year       = {2008},
        editor     = {Aluizio Leite and
                      Ana Lima Cecilio and Heloisa Jahn},
        editortype = {organizer},
        edition    = {2},
        publisher  = {Nova Fronteira},
        location   = {Rio de Janeiro},
        series     = {Biblioteca luso-brasileira.
                      Série brasileira},
    }
\end{verbatim}

\noindent
Ou:

\begin{verbatim}
    @inbook{bosi08,
        crossref = {assis08:1},
        title    = {Uma figura machadiana},
        author   = {Alfredo Bosi},
        pages    = {179-189},
    }
\end{verbatim}

\singlecite{bosi08}

\subsection{@suppbook}

Uma parte suplementar de um livro, com um título genérico, como ``prefácio'' ou ``introdução'':

\begin{verbatim}
    @suppbook{leite08,
        title      = {Nota Editorial},
        author     = {Aluizio Leite and
                      Ana Lima Cecilio and Heloisa Jahn},
        pages      = {1-5},
        volume     = {1},
        booktitle  = {Fortuna crítica/Romance},
        pagetotal  = {1340},
        bookauthor = {Machado de Assis},
        maintitle  = {Obra completa em quatro volumes},
        year       = {2008},
        editor     = {Aluizio Leite and
                      Ana Lima Cecilio and Heloisa Jahn},
        editortype = {organizer},
        edition    = {2},
        publisher  = {Nova Fronteira},
        location   = {Rio de Janeiro},
        series     = {Biblioteca luso-brasileira.
                      Série brasileira},
    }
\end{verbatim}

\noindent
Ou:

\begin{verbatim}
    @suppbook{leite08,
        crossref = {assis08:1},
        title    = {Nota Editorial},
        author   = {Aluizio Leite and
                    Ana Lima Cecilio and Heloisa Jahn},
        pages    = {1-5},
    }
\end{verbatim}

\singlecite{leite08}

\subsection{@mvcollection}

Uma coleção abrangendo diversos volumes, cada um composto por diversas contribuições independentes, com seus próprios autores e títulos. A obra como um todo não possui um autor, mas geralmente possui um editor:

\begin{verbatim}
    @mvcollection{moretti09,
        editor     = {Franco Moretti},
        editortype = {organizer},
        translator = {Denise Bottmann},
        title      = {O Romance},
        volumes    = {5},
        publisher  = {Cosac Naify},
        location   = {São Paulo},
        year       = {2009},
    }
\end{verbatim}

\singlecite{moretti09}

\subsection{@collection}

Uma única coleção composta por diversas contribuições independentes. Pode ser um dos volumes de uma coleção que abrange múltiplos volumes:

\begin{verbatim}
    @collection{moretti09:1,
        volume      = {1},
        title       = {A cultura do romance},
        pagetotal   = {1120},
        illustrated = {40 ils.},
        editor      = {Franco Moretti},
        editortype  = {organizer},
        translator  = {Denise Bottmann},
        maintitle   = {O Romance},
        publisher   = {Cosac Naify},
        location    = {São Paulo},
        year        = {2009},
    }
\end{verbatim}

\noindent
Ou:

\begin{verbatim}
    @collection{moretti09:1,
        crossref    = {moretti09},
        volume      = {1},
        title       = {A cultura do romance},
        pagetotal   = {1120},
        illustrated = {40 ils.},
    }
\end{verbatim}

\singlecite{moretti09:1}

\subsection{@incollection}

Uma contribuição a uma coleção, formando uma unidade independente com autor e título próprios:

\begin{verbatim}
    @incollection{mann09,
        author      = {Thomas Mann},
        title       = {Bilse e eu},
        pages       = {217},
        volume      = {1},
        booktitle   = {A cultura do romance},
        pagetotal   = {1120},
        illustrated = {40 ils.},
        editor      = {Franco Moretti},
        editortype  = {organizer},
        translator  = {Denise Bottmann},
        maintitle   = {O Romance},
        publisher   = {Cosac Naify},
        location    = {São Paulo},
        year        = {2009},
    }
\end{verbatim}

\noindent
Ou:

\begin{verbatim}
    @incollection{mann09,
        crossref = {moretti09:1},
        author   = {Thomas Mann},
        title    = {Bilse e eu},
        pages    = {217},
    }
\end{verbatim}

\singlecite{mann09}

\subsection{@suppcollection}

Uma parte suplementar de uma coleção, com um título genérico, como ``prefácio'' ou ``introdução'':

\begin{verbatim}
    @suppcollection{moretti09:1b,
        title       = {Apresentação geral},
        author      = {Franco Moretti},
        pages       = {217},
        volume      = {1},
        booktitle   = {A cultura do romance},
        pagetotal   = {1120},
        illustrated = {40 ils.},
        editor      = {Franco Moretti},
        editortype  = {organizer},
        translator  = {Denise Bottmann},
        maintitle   = {O Romance},
        publisher   = {Cosac Naify},
        location    = {São Paulo},
        year        = {2009},
    }
\end{verbatim}

\noindent
Ou:

\begin{verbatim}
    @suppcollection{moretti09:1b,
        crossref = {moretti09:1},
        title    = {Apresentação geral},
        author   = {Franco Moretti},
    }
\end{verbatim}

\singlecite{moretti09:1b}

\subsection{@article}

Um artigo científico/acadêmico:

\begin{verbatim}
    @article{negrão14,
        title    = {Brazilian Portuguese as a
                    transatlantic language},
        subtitle = {agents of linguistic contact},
        author   = {Esmeralda Vailati Negrão and Evani Viotti},
        journal  = {Interdisciplinary Journal of
                    Portuguese Diaspora Studies},
        volume   = {3},
        pages    = {135-154},
        year     = {2014},
    }
\end{verbatim}

\singlecite{negrão14}

\subsection{@thesis}

Uma dissertação de mestrado:

\begin{verbatim}
    @thesis{eliseu84,
        title       = {Verbos ergativos do Português},
        subtitle    = {descrição e análise},
        author      = {André Manuel Godinho Simões Eliseu},
        type        = {Dissertação (Mestrado em Linguística)},
        institution = {Universidade de Lisboa},
        location    = {Lisboa},
        eventyear   = {1985},
    }
\end{verbatim}

\singlecite{eliseu84}

Uma tese de doutorado:

\begin{verbatim}
    @thesis{amaral15,
        title       = {A alternância transitivo-intransitiva
                       no português brasileiro},
        subtitle    = {fenômenos semânticos},
        author      = {Luana Lopes Amaral},
        type        = {Tese (Doutorado em Linguística)},
        institution = {Universidade Federal de Minas Gerais},
        location    = {Belo Horizonte},
        eventyear   = {2015},
    }
\end{verbatim}

\singlecite{amaral15}

\subsection{@inproceedings}

Resumos ou anais de eventos:

\begin{verbatim}
    @inproceedings{negrão13,
        title      = {A emergência da sintaxe do
                      português brasileiro},
        subtitle   = {absolutas, alçamento do
                      possuidor e passivas},
        author     = {Esmeralda Vailati Negrão and Evani Viotti},
        eventtitle = {Encontro nacional do gt de
                      teoria da gramática da ANPOLL},
        number     = {28},
        venue      = {Florianópolis},
        eventyear  = {2013},
        booktitle  = {Caderno de Resumos},
        publisher  = {ANPOLL},
        location   = {Campinas},
        year       = {2013},
    }
\end{verbatim}

\singlecite{negrão13}


\clearpage
\section{Compatibilidade com o pacote \texttt{abntex2cite}}

Em geral é possível usar o mesmo arquivo \texttt{.bib} utilizado pelo \texttt{abntex2cite}, e para a maior parte das entradas mais simples nenhuma mudança é necessária; estas são as principais exceções:

\begin{itemize}
\begin{sloppypar}
    \item Não consegui entender por que, mas os exemplos do \texttt{abntex2cite} usam \texttt{{\'\i}} para o caractere ``í'', enquanto o normal me parece ser \texttt{{\'i}} (para outras letras com acento agudo o \texttt{abntex2cite} usa o formato normal). Isso pode causar alguns problemas, o ideal é usar \texttt{{\'i}} (ou aproveitar os recursos do biber e usar a codificação utf-8). Caso isso não seja possível por algum motivo, chamar o biblatex com a opção \texttt{safeinputenc} pode resolver alguns problemas (mas também causa alguns outros, interferindo, por exemplo, na capitalização automática de palavras acentuadas).
    \item Os campos de datas (\texttt{date}, \texttt{year}, \texttt{month} etc.) devem seguir o formato usado pelo \texttt{biblatex}, \texttt{yyyy-mm-dd}. Cf. seção 2.2.1 do manual.
    \item O \texttt{biblatex} diferencia os campos \texttt{pages} (e.g. ``p. 12-18'') e \texttt{pagetotal} (e.g. ``347 p.'').
    \item Quando o primeiro campo impresso é o título, a primeira palavra é automaticamente impressa em maiúsculas. Para que mais de uma palavra seja impressa em maiúsculas, coloque-as entre chaves ou separadas por um no breaking space (\texttt{~}).
    \item Quando o primeiro campo impresso é a organização, todo o campo é impresso em maiúsculas. Para que a capitalização de uma palavra não seja alterada, coloque-a entre chaves (e.g. \texttt{{pAlaVRa}}).
    \item Pode-se incluir no campo \texttt{options} as opções \texttt{nosl} (para não mostrar a abreviação ``[s.l]''), \texttt{nosn} (para não mostrar ``[s.n.]'') ou \texttt{noslsn} (para não mostrar nenhuma das duas abreviações) (e.g. \texttt{options = {noslsn}}); essa configuração tem efeito em cada entrada específica. Pode-se também incluir essas opções ao chamar o \texttt{biblatex} para nunca mostrar as abreviações, e então abrir exceções para entradas específicas (e.g. \texttt{options = {noslsn=false}}).
    \item Para periódicos, deve se utilizar entradas \texttt{@periodical}. Usando-se \texttt{@book}, como no \texttt{abntex2cite}, o ISSN não aparece.
    \item Para teses de mestrado e doutorado, o campo \texttt{pagetotal} é automaticamente formatado em folhas, como requer a norma (e.g. ``347 f.''). Para usar páginas também nessas entradas, use \texttt{bookpagination = {page}}. Para usar folhas em outras entradas, use \texttt{bookpagination = {sheet}}.
    \item Ao modificar o campo \texttt{type} em teses de mestrado e doutorado, deve-se preenchê-lo com todo o texto desejado (e.g. ``Tese (Doutorado em Nutrição)'').
    \item Ao utilizar o campo \texttt{illustrated}, deve-se preenchê-lo com todo o texto desejado (e.g. ``il.'').
    \item Além do campo \texttt{organization}, há o campo \texttt{nameaddon}, de modo que organizações como ``BRASIL. Supremo Tribunal de Justiça'' e ``BRASIL. Supremo Tribunal Federal'' podem ter uma mesma organização, diferenciando-se por esse campo. Isso permite que a primeira parte, ``BRASIL'', não seja repetida várias vezes seguidas (e.g. ``\underline{\hspace*{4em}}. Supremo Tribunal Federal'').
    \item O \texttt{abntex2cite} às vezes usa o campo \texttt{type} como um complemento da organização (um exemplo é a entrada \texttt{brasil1988}). Com o \texttt{biblatex-abnt} deve-se usar o campo \texttt{nameaddon} em vez de \texttt{type}. Embora nesse caso (``Constituição (1988)'') o campo \texttt{type} faça mais sentido semanticamente, colocá-lo nessa posição causaria problemas na organização alfabética da bibliografia, já que o campo \texttt{type} às vezes apareceria antes do título e às vezes depois.
    \item A recomendação do manual do \texttt{biblatex} (seção 2.3.3) é de que, para autores corporativos, utilize-se os campos \texttt{author} e \texttt{editor}, colocando o nome da organização entre chaves. Essa opção tem a vantagem de permitir que se misture autores corporativos e autores comuns (e.g. \texttt{editor = {{National} \texttt{Aeronautics and Space Administration} and Doe, John}}).
    \item Quando o nome do autor, editor ou organização for muito grande para usar nas citações, pode-se acrescentar os campos \texttt{short\-author} e \texttt{short\-editor} (e.g. \texttt{author = {National Aeronautics and Space Ad\-min\-is\-tra\-tion},} \texttt{short\-author = {NASA}} imprimirá ``NASA'' nas citações e ``National Aeronautics and Space Administration'' na bibliografia). O campo \texttt{org-short}, usado pelo \texttt{abntex\-2\-cite}, é automaticamente convertido para \texttt{shortauthor}.
    \item O separador ``de'' faz com que tudo o que vem depois dele seja considerado um único sobrenome (e.g. na entrada \texttt{alves1995} o nome \texttt{Roque de Brito} \texttt{Alves} é impresso como ``BRITO ALVES, Roque de''). Cf. \url{http://tex.stackexchange.com/q/308625/102699}.
    \item Quando os campos \texttt{number}, \texttt{volume}, \texttt{chapter} e \texttt{edition} contém apenas números, eles são formatados automaticamente (e.g. \texttt{edition = {5}} imprime ``5. ed.''). Quando esses campos contém letras, deve-se preencher todo o conteúdo desejado (e.g. \texttt{edition = {5th. ed}} imprime ``5th. ed.''). Os caracteres \texttt{.,-/} podem ser usados e o campo ainda será considerado como contendo apenas números.
    \item Em entradas dos tipos \texttt{phdthesis}, \texttt{mastersthesis} e \texttt{monography}, há a data de publicação, que aparece logo após o título, e a data da defesa, que aparece por último. O \texttt{abntex2cite} às vezes usa o campo \texttt{year-presented} para diferenciar entre as duas datas e às vezes muda as opções do pacote para mostrar uma mesma data em uma ou outra posição. No \texttt{biblatex-abnt} pode-se usar os campos de datas usuais para a data de publicação, que aparece após o título, e os campos \texttt{eventdate}, \texttt{eventmonth} e \texttt{eventyear} para a data da defesa, que aparece no final. O campo \texttt{year-presented}, usado pelo \texttt{abntex2cite}, é automaticamente convertido para \texttt{eventyear}. (Cf. entradas \texttt{morgado1990}, \texttt{morgadob1990} e \texttt{morgadoc1990} no arquivo \texttt{abnt-testcase.tex}.)
    \item Em vez dos campos \texttt{reprinted-from} e \texttt{reprinted-text}, utilizados pelo \texttt{abntex2cite}, usa-se o campo \texttt{related} para citar uma entrada relacionada e o campo \texttt{relatedtype} para especificar a natureza dessa relação. O campo \texttt{reprinted-from} é automaticamente convertido para \texttt{related}; esse campo deve conter a chave da obra relacionada. O campo \texttt{relatedtype} pode conter algumas opções: \texttt{relatedtype={reprintfrom}} imprime ``Separata de'' (esse texto também é usado como padrão quando se escreve qualquer coisa no campo \texttt{reprinted-text}, usado no \texttt{abntex2cite}); \texttt{recensionof} imprime ``Recensão de''; \texttt{reviewof} imprime ``Resenha de''; \texttt{reprintof} imprime ``Reimpressão de''; \texttt{translationof} imprime ``Tradução de''. Outras possibilidades podem ser encontradas no arquivo \texttt{brazilian.lbx}, incluso na instalação padrão do \texttt{biblatex}.
    \item As entradas \texttt{inbook} e \texttt{incollection}, no \texttt{biblatex}, não se comportam como no \texttt{bibtex}. Entradas \texttt{inbook} também podem ter um \texttt{bookauthor}; elas estão para as entradas \texttt{book} assim como as entradas \texttt{incollection} estão para as entradas \texttt{collection} (Cf. seção 2.3.1 do manual do \texttt{biblatex}). No \texttt{biblatex-abnt} é possível definir um \texttt{bookauthor} para entradas \texttt{inbook}, mas, caso isso não seja feito, essas entradas se comportarão como as entradas \texttt{inbook} do \texttt{bibtex}.
    \item É possível usar o campo \texttt{furtherresp} como no \texttt{abntex2cite}, mas é preferível usar os campos \texttt{editora}, \texttt{editoratype}, \texttt{editorb}, \texttt{editorbtype} etc. (e.g. \texttt{editora = {Ismael Cardim},  editoratype={coeditor}} imprimirá ``Co-edição de Ismael Cardim'' na bibliografia; cf. entradas \texttt{hou\-a\-iss\-1996}, \texttt{koogan1998}, \texttt{ceravi1983}, \texttt{riofilme1998} e pesquise pelo campo \texttt{editor\-a\-type} para mais exemplos). Todas as opções para o campo \texttt{editor\-type} podem ser encontradas no arquivo \texttt{abnt-brazilian.lbx}. Usar esses campos em lugar do campo \texttt{furtherresp} assegurará que as entradas sejam impressas de forma consistente, embora as entradas da própria ABNT não o sejam.
    \item Para tradutores pode-se usar o campo \texttt{translator}.
\end{sloppypar}
\end{itemize}

\clearpage
\nocite{*}
\printbibliography

\end{document}
